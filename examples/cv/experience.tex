%-------------------------------------------------------------------------------
%	SECTION TITLE
%-------------------------------------------------------------------------------
\cvsection{Experience}


%-------------------------------------------------------------------------------
%	CONTENT
%-------------------------------------------------------------------------------
\begin{cventries}

%---------------------------------------------------------
  \cventry
    {Founding Member \& Site Reliability Engineer \& Infrastructure Team Lead} % Job title
    {Danggeun Pay Inc. (KarrotPay)} % Organization
    {Seoul, S.Korea} % Location
    {Mar. 2021 - Present} % Date(s)
    {
      \begin{cvitems} % Description(s) of tasks/responsibilities
        \item {Everything that matters.}
        \item {Designed and provisioned the entire infrastructure on the AWS cloud to meet security compliance and acquire a business license for financial services in Korea.}
        \item {Continuously improved the infrastructure architecture since launching the service. (currently 3.6 million users)}
        \item {Established a standardized base for declarative management of infrastructures and service deployments, enabling operational efficiency and consistency. Over 90\% of AWS resources were all managed through standardized terraform modules. All add-ons and service workloads on the Kubernetes cluster were managed on a GitOps basis with Kustomize and ArgoCD.}
        \item {Saved over 30\% of the overall AWS costs by establishing a quarterly purchasing strategiy for RI (Reserved Instance) and SP (Savings Plan) and by introducing Graviton instances.}
        \item {Established a core architecture for regulating of outbound DNS traffic in multi-account and multi-VPC environments utilizing AWS Route53 DNS Firewall and FMS. This significantly increased the level of security confidence in the financial sector's segregated environment.}
        \item {Introduced Okta employee identity solution in the company, establishing security policies and configuring SSO integration with over 20 enterprise systems including AWS, GitHub, Slack, Google Workspace. Set up a Hub and Spoke architecture, enabling a collaborative account structure with the parent company, Daangn Market.}
      \end{cvitems}
    }

%---------------------------------------------------------
  \cventry
    {Site Reliability Engineer} % Job title
    {Danggeun Market Inc.} % Organization
    {Seoul, S.Korea} % Location
    {Feb. 2021 - Mar. 2021} % Date(s)
    {
    }

%---------------------------------------------------------
  \cventry
    {Founding Member \& Director of Infrastructure Division} % Job title
    {Kasa} % Organization
    {Seoul, S.Korea} % Location
    {Jun. 2018 - Jan. 2021} % Date(s)
    {
      \begin{cvitems} % Description(s) of tasks/responsibilities
        \item {Designed on-boarding process to guide new engineers, help them to focus on the right tasks, and set expectations to help them be successful at Infrastructure team.}
        \item {Migrated the orchestration system from DC/OS to Kubernetes which is based on AWS EKS. Managed 3 Kubernetes clusters and 300+ pods. Managed all Kubernetes manifests declaratively with Kustomize and ArgoCD.}
        \item {Designed and managed complex network configurations on AWS with 4 VPC and 100+ subnets. Separated the development network and operation network according to financial regulations. Established dedicated network connections from AWS VPC to partners' on-premise network based on AWS Direct Connect with secure connection using IPsec VPN. Provisioned OpenVPN servers with LDAP integration.}
        \item {Provisioned a observability system with Kafka, Elastic Stack(Filebeat, Heartbeat, APM Server, Logstash, Elasticsearch, Kibana). Collected log, uptime, tracing data from hosts, containers, pods and more. The ES cluster which has 9 nodes processed more than 1 billion documents per month. Wrote Terraform module to easily provision ES cluster on AWS EC2 instances.}
        \item {Provisioned a monitoring system with Kafka, Telegraf, InfluxDB, Grafana. Collected metrics from hosts, containers, pods and more. Wrote Terraform module to easily provision InfluxDB with HA on AWS EC2 instances.}
        \item {Introduced Kong API Gateway to easily connect all API microservices with a declarative management method based on Terraform and Atlantis to collaborate and audit change history.}
        \item {Provisioned the Directory Service for employee identity management based on OpenLDAP which guarantees HA with multi-master replication.}
        \item {Implemented Worker microservices consuming Kafka event topics for email, SMS, Kakaotalk and Slack notification. Developed in-house framework to easily build Kafka consumer microservice with common features including retry on failure, DLQ(Dead Letter Queue), event routing and more.}
        \item {Introduced Elastic APM to help distributed tracing, trouble-shooting and performance testing in MSA.}
      \end{cvitems}
    }

%---------------------------------------------------------
  \cventry
    {Administrador de sistemas (practicas de empresa)} % Job title
    {ASAC} % Organization
    {Asturias, España} % Location
    {Feb.. 2021 - Mar. 2021} % Date(s)
    {
      \begin{cvitems} % Description(s) of tasks/responsibilities
        \item {Practicas de empresa en un Datacenter Tien 3. Trabajos con hardware especifico del servidor.}
        \item {Documentacion de los procedimientos de actuacion y trato con los clientes.}
      \end{cvitems}
    }

%---------------------------------------------------------
  \cventry
    {Becario de colaboración de servicios informaticos} % Job title
    {PLAT Corp.} % Organization
    {Seoul, S.Korea} % Location
    {Jan. 2016 - Jun. 2017} % Date(s)
    {
      \begin{cvitems} % Description(s) of tasks/responsibilities
        \item {Implemented RESTful API server for car rental booking application(CARPLAT in Google Play).}
        \item {Built and deployed overall service infrastructure utilizing Docker container, CircleCI, and several AWS stack(Including EC2, ECS, Route 53, S3, CloudFront, RDS, ElastiCache, IAM), focusing on high-availability, fault tolerance, and auto-scaling.}
        \item {Developed an easy-to-use Payment module which connects to major PG(Payment Gateway) companies in Korea.}
      \end{cvitems}
    }

%---------------------------------------------------------
\end{cventries}
